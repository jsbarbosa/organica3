%%%%%%%%%%%%%%%%%%%%%%%%%%%%%%%%%%%%%%%%%
% Stylish Article
% LaTeX Template
% Version 2.1 (1/10/15)
%
% This template has been downloaded from:
% http://www.LaTeXTemplates.com
%
% Original author:
% Mathias Legrand (legrand.mathias@gmail.com) 
% With extensive modifications by:
% Vel (vel@latextemplates.com)
% Final ACS by:
% Juan Barbosa
% License:
% CC BY-NC-SA 3.0 (http://creativecommons.org/licenses/by-nc-sa/3.0/)
%
%%%%%%%%%%%%%%%%%%%%%%%%%%%%%%%%%%%%%%%%%
\documentclass[fleqn,10pt]{SelfArx}
%\usepackage[superscript]{cite}
\usepackage{wrapfig}
%----------------------------------------------------------------------------------------
%	ARTICLE INFORMATION
%----------------------------------------------------------------------------------------

\JournalInfo{Laboratorio Org\'anica 3, No. 1, 18/08/2017} % Journal information
\Archive{ }

\PaperTitle{Reacci\'on de Barbier} %
%\Keywords{Keyword1 --- Keyword2 --- Keyword3} % Keywords - if you don't want any simply remove all the text between the curly brackets
%\newcommand{\keywordname}{Keywords} % Defines the keywords heading name

%----------------------------------------------------------------------------------------
%	ABSTRACT
%----------------------------------------------------------------------------------------

\Abstract{}

%----------------------------------------------------------------------------------------

\begin{document}

\flushbottom % Makes all text pages the same height

\maketitle % Print the title and abstract box

%\tableofcontents % Print the contents section

\thispagestyle{empty} % Removes page numbering from the first page



%----------------------------------------------------------------------------------------
%	ARTICLE CONTENTS
%----------------------------------------------------------------------------------------

\section*{Introducci\'on} % The \section*{} command stops section numbering
%------------------------------------------------

\section{Resultados y Discusi\'on}

\section{Conclusiones}
\section{Secci\'on experimental}
Una soluci\'on de bromuro de alilo (26.0 mmol), yoduro de potasio (43.1 mmol) y cloruro de esta\~no dihidratado (25.8 mmol) se prepara usando 50 mL de agua destilada. Sobre esta soluci\'on se adiciona gota a gota una soluci\'on de glioxal al 4 \%. La reacci\'on se lleva a cabo en dos etapas, la primera a 38.4 $^\circ$C por 1 hora, porteriormente se retira del calor y se deja a temperatura ambiente y atm\'osfera inerte por otras 18 horas.

El producto se extrae con acetato de etilo, se adiciona \'acido clorh\'idrico (1M) y salmuera. El extracto se filtra en sulfato de sodio y posteriormente en celita. El solvente se evapora por rotaevaporaci\'on, y el producto se lava en cloroformo.

%----------------------------------------------------------------------------------------
%	REFERENCE LIST
%----------------------------------------------------------------------------------------
\phantomsection
\bibliography{informe}
\bibliographystyle{unsrt}

%----------------------------------------------------------------------------------------
\newpage
\onecolumn
\section{Informaci\'on suplementaria}\label{sec: complementaria}
\end{document}